% !TEX TS-program = lualatex
% !TEX encoding = UTF-8

\documentclass[11pt, twoside]{report}

\usepackage{fontspec}
\usepackage[utf8]{inputenc}
\usepackage[bitstream-charter]{mathdesign}
\usepackage{bbding}
\usepackage{ragged2e}
\usepackage{parskip}
\usepackage{enumitem}
\usepackage{titlesec}
\usepackage{paracol}
\usepackage{mdframed}
\usepackage[margin=1in]{geometry}

\usepackage[autocompile]{gregoriotex}

\titleformat{\chapter}[block]{\huge\scshape\filcenter}{}{1em}{}
\titleformat{\section}[block]{\Large\bfseries\filcenter}{}{1em}{}

\mdfsetup{skipabove=\topskip, skipbelow=\topskip}

\newenvironment{versicles}{\par\leavevmode\parskip=0pt}{}

\begin{document}

\chapter*{Lauds in Paschaltide}

\textit{All make the sign of the cross with the officiant as he begins:}

\gresetinitiallines{1}
\gregorioscore{../deus-in-adjutorium}

\textit{On ferias, the following tone is used:}

\gresetinitiallines{0}
\gregorioscore{../deus-in-adjutorium-ferial}

\section*{Psalm 92}

\textit{The antiphons are doubled on Low Sunday.}

\gresetinitiallines{1}
\gregorioscore{ps92}

% c4
% 8G
% 1. Dó(g)mi(h)nus(j) re(j)gná(j)vit,(j) de(j)có(j)rem(j) ind(j)<b>ú</b>(k)tus(jr) est:(j.) *(:) ind(j)ú(j)tus(j) est(j) Dó(j)mi(j)nus(j) for(j)ti(j)tú(j)di(j)nem,(j) <i>et</i>(i) <i>præ</i>(j)<b>cín</b>(h)xit(gr) se.(g.) (::)
% 8G*
% 1. Dó(g)mi(h)nus(j) re(j)gná(j)vit,(j) de(j)có(j)rem(j) ind(j)<b>ú</b>(k)tus(jr) est:(j.) *(:) ind(j)ú(j)tus(j) est(j) Dó(j)mi(j)nus(j) for(j)ti(j)tú(j)di(j)nem,(j) <i>et</i>(i) <i>præ</i>(j)<b>cín</b>(h)xit(gr) se.(gh..) (::)
% 8c
% 1. Dó(g)mi(h)nus(j) re(j)gná(j)vit,(j) de(j)có(j)rem(j) ind(j)<b>ú</b>(k)tus(jr) est:(j.) *(:) ind(j)ú(j)tus(j) est(j) Dó(j)mi(j)nus(j) for(j)ti(j)tú(j)di(j)nem,(j) <i>et</i>(h) <i>præ</i>(j)<b>cín</b>(k)xit(jr) se.(j.) (::)

2. Etenim firmávit orbem \textbf{ter}ræ,~* qui non \textit{com}\textit{mo}\textbf{vé}bitur.

3. Paráta sedes tua \textbf{ex} tunc:~* a s\'{\ae}\textit{cu}\textit{lo} \textbf{tu} es.

4. Elevavérunt flúmina, \textbf{Dó}mine:~* elevavérunt flúmina \textit{vo}\textit{cem} \textbf{su}am.

5. Elevavérunt flúmina fluctus \textbf{su}os,~* a vócibus aquá\textit{rum} \textit{mul}\textbf{tá}rum.

6. Mirábiles elatiónes \textbf{ma}ris:~* mirábilis in \textit{al}\textit{tis} \textbf{Dó}minus.

7. Testimónia tua credibília facta sunt \textbf{ni}mis:~* domum tuam decet sanctitúdo, Dómine, in longitúdi\textit{nem} \textit{di}\textbf{é}rum.

8. Glória Patri, et \textbf{Fí}lio,~* et Spirí\textit{tu}\textit{i} \textbf{Sanc}to.

9. Sicut erat in princípio, et nunc, et \textbf{sem}per,~* et in s\'{\ae}cula sæcu\textit{ló}\textit{rum}. \textbf{A}men.

\section*{Psalm 99}

\gresetinitiallines{0}
\gregorioscore{ps99}

% c4

2. Introíte in conspéctu \textbf{e}jus,~* in exsul\textit{ta}\textit{ti}\textbf{ó}ne.

3. Scitóte quóniam Dóminus ipse est \textbf{De}us:~* ipse fecit nos, \textit{et} \textit{non} \textbf{ip}si nos.

4. Pópulus ejus, et oves páscuæ ejus:~\GreDagger\ introíte portas ejus in confessi\textbf{ó}ne,~* átria ejus in hymnis: confité\textit{mi}\textit{ni} \textbf{il}li.

5. Laudáte nomen ejus: quóniam suávis est Dóminus,~\GreDagger\ in ætérnum misericórdia \textbf{e}jus,~* et usque in generatiónem et generatiónem vé\textit{ri}\textit{tas} \textbf{e}jus.

6. Glória Patri, et \textbf{Fí}lio,~* et Spirí\textit{tu}\textit{i} \textbf{Sanc}to.

7. Sicut erat in princípio, et nunc, et \textbf{sem}per,~* et in s\'{\ae}cula sæcu\textit{ló}\textit{rum}. \textbf{A}men.

\section*{Psalms 62 \& 66}

\gresetinitiallines{0}
\gregorioscore{ps62}

% c4
% 8G
% 1. De(g)us,(h) De(j)us(j) <b>me</b>(k)us,(j.) *(:) ad(j) te(j) de(j) <i>lu</i>(i)<i>ce</i>(j) <b>ví</b>(h)gi(g)lo.(g.) <i>Flex:</i>(::)  in(j)a(j)quó(j)sa: +(h. j j j  ::)
% 8G*
% 1. De(g)us,(h) De(j)us(j) <b>me</b>(k)us,(j.) *(:) ad(j) te(j) de(j) <i>lu</i>(i)<i>ce</i>(j) <b>ví</b>(h)gi(g)lo.(gh..) <i>Flex:</i>(::)  in(j)a(j)quó(j)sa: +(h. j j j  ::)
% 8c
% 1. De(g)us,(h) De(j)us(j) <b>me</b>(k)us,(j.) *(:) ad(j) te(j) de(j) <i>lu</i>(h)<i>ce</i>(j) <b>ví</b>(k)gi(j)lo.(j.) <i>Flex:</i>(::)  in(j)a(j)quó(j)sa: +(h. j j j  ::)

2. Sitívit in te ánima \textbf{me}a,~* quam multiplíciter tibi \textit{ca}\textit{ro} \textbf{me}a.

3. In terra desérta, et ínvia, et inaquósa:~\GreDagger\ sic in sancto appárui \textbf{ti}bi,~* ut vidérem virtútem tuam, et gló\textit{ri}\textit{am} \textbf{tu}am.

4. Quóniam mélior est misericórdia tua super \textbf{vi}tas:~* lábia me\textit{a} \textit{lau}\textbf{dá}bunt te.

5. Sic benedícam te in vita \textbf{me}a:~* et in nómine tuo levábo \textit{ma}\textit{nus} \textbf{me}as.

6. Sicut ádipe et pinguédine repleátur ánima \textbf{me}a:~* et lábiis exsultatiónis laudá\textit{bit} \textit{os} \textbf{me}um.

7. Si memor fui tui super stratum meum,~\GreDagger\ in matutínis meditábor \textbf{in} te:~* quia fuísti ad\textit{jú}\textit{tor} \textbf{me}us.

8. Et in velaménto alárum tuárum exsultábo,~\GreDagger\ adh\'{\ae}sit ánima mea \textbf{post} te:~* me suscépit déx\textit{te}\textit{ra} \textbf{tu}a.

9. Ipsi vero in vanum quæsiérunt ánimam meam,~\GreDagger\ introíbunt in inferióra \textbf{ter}ræ:~* tradéntur in manus gládii, partes vúl\textit{pi}\textit{um} \textbf{e}runt.

10. Rex vero lætábitur in Deo,~\GreDagger\ laudabúntur omnes qui jurant in \textbf{e}o:~* quia obstrúctum est os loquénti\textit{um} \textit{in}\textbf{í}qua.

\textbf{Ps. 66.} Deus misereátur nostri, et benedícat \textbf{no}bis:~* illúminet vultum suum super nos, et misere\textit{á}\textit{tur} \textbf{nos}tri.

2. Ut cognoscámus in terra viam \textbf{tu}am,~* in ómnibus Géntibus salu\textit{tá}\textit{re} \textbf{tu}um.

3. Confiteántur tibi pópuli, \textbf{De}us:~* confiteántur tibi pó\textit{pu}\textit{li} \textbf{om}nes.

4. Læténtur et exsúltent \textbf{Gen}tes:~* quóniam júdicas pópulos in æquitáte, et Gentes in \textit{ter}\textit{ra} \textbf{dí}rigis.

5. Confiteántur tibi pópuli, Deus, confiteántur tibi pópuli \textbf{om}nes:~* terra dedit \textit{fruc}\textit{tum} \textbf{su}um.

6. Benedícat nos Deus, Deus noster, benedícat nos \textbf{De}us:~* et métuant eum omnes \textit{fi}\textit{nes} \textbf{ter}ræ.

7. Glória Patri, et \textbf{Fí}lio,~* et Spirí\textit{tu}\textit{i} \textbf{Sanc}to.

8. Sicut erat in princípio, et nunc, et \textbf{sem}per,~* et in s\'{\ae}cula sæcu\textit{ló}\textit{rum}. \textbf{A}men.

\gresetinitiallines{0}
\gregorioscore{ana-1}

\section*{The Canticle of the Three Children}

\gresetinitiallines{1}
\gregorioscore{benedicite}

% c4
% 6F
% 1. Be(f)ne(gh)dí(h)ci(h)te,(h) óm(h)ni(h)a(h) ó(h)pe(h)ra(h) Dó(h)mi(h)<i>ni</i>,(g) <b>Dó</b>(h)mi(fr)no:(f.) *(:) lau(h)dá(h)te(h) et(h) su(h)per(h)ex(h)al(h)tá(h)te(h) e(h)<i>um</i>(f) <i>in</i>(gh) <b>s<sp>'ae</sp></b>(g)cu(fr)la.(f.) (::)

2. Benedícite, Ángeli Dómi\textit{ni}, \textbf{Dó}mino:~* benedícite, \textit{cæ}\textit{li}, \textbf{Dó}mino.

3. Benedícite, aquæ omnes, quæ super cælos \textit{sunt}, \textbf{Dó}mino:~* benedícite, omnes virtútes Dó\textit{mi}\textit{ni}, \textbf{Dó}mino.

4. Benedícite, sol et lu\textit{na}, \textbf{Dó}mino:~* benedícite, stellæ \textit{cæ}\textit{li}, \textbf{Dó}mino.

5. Benedícite, omnis imber et \textit{ros}, \textbf{Dó}mino:~* benedícite, omnes spíritus \textit{De}\textit{i}, \textbf{Dó}mino.

6. Benedícite, ignis et æs\textit{tus}, \textbf{Dó}mino:~* benedícite, frigus et \textit{æs}\textit{tus}, \textbf{Dó}mino.

7. Benedícite, rores et pruí\textit{na}, \textbf{Dó}mino:~* benedícite, gelu et \textit{fri}\textit{gus}, \textbf{Dó}mino.

8. Benedícite, glácies et ni\textit{ves}, \textbf{Dó}mino:~* benedícite, noctes et \textit{di}\textit{es}, \textbf{Dó}mino.

9. Benedícite, lux et téne\textit{bræ}, \textbf{Dó}mino:~* benedícite, fúlgura et \textit{nu}\textit{bes}, \textbf{Dó}mino.

10. Benedícat ter\textit{ra} \textbf{Dó}minum:~* laudet et superexáltet e\textit{um} \textit{in} \textbf{s\'{\ae}}cula.

11. Benedícite, montes et col\textit{les}, \textbf{Dó}mino:~* benedícite, univérsa germinántia in \textit{ter}\textit{ra}, \textbf{Dó}mino.

12. Benedícite, fon\textit{tes}, \textbf{Dó}mino:~* benedícite, mária et flú\textit{mi}\textit{na}, \textbf{Dó}mino.

13. Benedícite, cete, et ómnia, quæ movéntur in a\textit{quis}, \textbf{Dó}mino:~* benedícite, omnes vólucres \textit{cæ}\textit{li}, \textbf{Dó}mino.

14. Benedícite, omnes béstiæ et péco\textit{ra}, \textbf{Dó}mino:~* benedícite, fílii hó\textit{mi}\textit{num}, \textbf{Dó}mino.

15. Benedícat Isra\textit{ël} \textbf{Dó}minum:~* laudet et superexáltet e\textit{um} \textit{in} \textbf{s\'{\ae}}cula.

16. Benedícite, sacerdótes Dómi\textit{ni}, \textbf{Dó}mino:~* benedícite, servi Dó\textit{mi}\textit{ni}, \textbf{Dó}mino.

17. Benedícite, spíritus, et ánimæ justó\textit{rum}, \textbf{Dó}mino:~* benedícite, sancti, et húmiles \textit{cor}\textit{de}, \textbf{Dó}mino.

18. Benedícite, Ananía, Azaría, Mísa\textit{ël}, \textbf{Dó}mino:~* laudáte et superexaltáte e\textit{um} \textit{in} \textbf{s\'{\ae}}cula.

19. Benedicámus Patrem et Fílium cum Sanc\textit{to} \textbf{Spí}ritu:~* laudémus et superexaltémus e\textit{um} \textit{in} \textbf{s\'{\ae}}cula.

20. Benedíctus es, Dómine, in firmamén\textit{to} \textbf{cæ}li:~* et laudábilis, et gloriósus, et superexaltá\textit{tus} \textit{in} \textbf{s\'{\ae}}cula.

\gresetinitiallines{0}
\gregorioscore{ana-2}

\section*{Psalms 148, 149, \& 150}

\gresetinitiallines{1}
\gregorioscore{ps148}

% c4
% 6F
% 1. Lau(f)dá(gh)te(h) Dó(h)mi(h)num(h) <i>de</i>(g) <b>cæ</b>(h)lis:(f.) *(:) lau(h)dá(h)te(h) e(h)um(h) <i>in</i>(f) <i>ex</i>(gh)<b>cél</b>(g)sis.(f.) <i>Flex:</i>(::)  vír(h)gi(g)nes, +(g. h h h  ::)

2. Laudáte eum, omnes Ange\textit{li} \textbf{e}jus:~* laudáte eum, omnes vir\textit{tú}\textit{tes} \textbf{e}jus.

3. Laudáte eum, sol \textit{et} \textbf{lu}na:~* laudáte eum, omnes stel\textit{læ} \textit{et} \textbf{lu}men.

4. Laudáte eum, cæli \textit{cæ}\textbf{ló}rum:~* et aquæ omnes, quæ super cælos sunt, laudent \textit{no}\textit{men} \textbf{Dó}mini.

5. Quia ipse dixit, \textit{et} \textbf{fac}ta sunt:~* ipse mandávit, \textit{et} \textit{cre}\textbf{á}ta sunt.

6. Státuit ea in ætérnum, et in s\'{\ae}cu\textit{lum} \textbf{s\'{\ae}}culi:~* præcéptum pósuit, et non \textit{præ}\textit{ter}\textbf{í}bit.

7. Laudáte Dóminum \textit{de} \textbf{ter}ra,~* dracónes, et om\textit{nes} \textit{a}\textbf{býs}si.

8. Ignis, grando, nix, glácies, spíritus pro\textit{cel}\textbf{lá}rum:~* quæ fáciunt \textit{ver}\textit{bum} \textbf{e}jus:

9. Montes, et om\textit{nes} \textbf{col}les:~* ligna fructífera, et \textit{om}\textit{nes} \textbf{ce}dri.

10. Béstiæ, et univér\textit{sa} \textbf{pé}cora:~* serpéntes, et vólu\textit{cres} \textit{pen}\textbf{ná}tæ:

11. Reges terræ, et om\textit{nes} \textbf{pó}puli:~* príncipes, et omnes jú\textit{di}\textit{ces} \textbf{ter}ræ.

12. Júvenes, et vírgines,~\GreDagger\ senes cum junióribus laudent no\textit{men} \textbf{Dó}mini:~* quia exaltátum est nomen e\textit{jus} \textit{so}\textbf{lí}us.

13. Conféssio ejus super cælum \textit{et} \textbf{ter}ram:~* et exaltávit cornu pó\textit{pu}\textit{li} \textbf{su}i.

14. Hymnus ómnibus sanc\textit{tis} \textbf{e}jus:~* fíliis Israël, pópulo appropin\textit{quán}\textit{ti} \textbf{si}bi.

\textbf{Ps. 149.} Cantáte Dómino cánti\textit{cum} \textbf{no}vum:~* laus ejus in ecclési\textit{a} \textit{sanc}\textbf{tó}rum.

2. Lætétur Israël in eo, qui fe\textit{cit} \textbf{e}um:~* et fílii Sion exsúltent in \textit{re}\textit{ge} \textbf{su}o.

3. Laudent nomen ejus \textit{in} \textbf{cho}ro:~* in týmpano, et psaltério \textit{psal}\textit{lant} \textbf{e}i.

4. Quia beneplácitum est Dómino in pópu\textit{lo} \textbf{su}o:~* et exaltábit mansuétos \textit{in} \textit{sa}\textbf{lú}tem.

5. Exsultábunt sancti \textit{in} \textbf{gló}ria:~* lætabúntur in cubí\textit{li}\textit{bus} \textbf{su}is.

6. Exaltatiónes Dei in gútture \textit{e}\textbf{ó}rum:~* et gládii ancípites in máni\textit{bus} \textit{e}\textbf{ó}rum.

7. Ad faciéndam vindíctam in na\textit{ti}\textbf{ó}nibus:~* increpatió\textit{nes} \textit{in} \textbf{pó}pulis.

8. Ad alligándos reges eórum in \textit{com}\textbf{pé}dibus:~* et nóbiles eórum in má\textit{ni}\textit{cis} \textbf{fér}reis.

9. Ut fáciant in eis judícium \textit{con}\textbf{scríp}tum:~* glória hæc est ómnibus \textit{sanc}\textit{tis} \textbf{e}jus.

\textbf{Ps. 150.} Laudáte Dóminum in sanc\textit{tis} \textbf{e}jus:~* laudáte eum in firmaménto vir\textit{tú}\textit{tis} \textbf{e}jus.

2. Laudáte eum in virtúti\textit{bus} \textbf{e}jus:~* laudáte eum secúndum multitúdinem magnitú\textit{di}\textit{nis} \textbf{e}jus.

3. Laudáte eum in so\textit{no} \textbf{tu}bæ:~* laudáte eum in psaltéri\textit{o}, \textit{et} \textbf{cí}thara.

4. Laudáte eum in týmpano, \textit{et} \textbf{cho}ro:~* laudáte eum in chor\textit{dis}, \textit{et} \textbf{ór}gano.

5. Laudáte eum in cýmbalis benesonántibus:~\GreDagger\ laudáte eum in cýmbalis jubila\textit{ti}\textbf{ó}nis:~* omnis spíritus \textit{lau}\textit{det} \textbf{Dó}minum.

6. Glória Patri, \textit{et} \textbf{Fí}lio,~* et Spirí\textit{tu}\textit{i} \textbf{Sanc}to.

7. Sicut erat in princípio, et nunc, \textit{et} \textbf{sem}per,~* et in s\'{\ae}cula sæcu\textit{ló}\textit{rum}. \textbf{A}men.

\gresetinitiallines{0}
\gregorioscore{ana-3}

\section*{Little Chapter}

\textit{All stand. The Little Chapter is proper on Sundays of Paschaltide. On ferias, it is as follows \textnormal{(Rom. 6:9-10)}:}

Christus resúrgens ex mórtuis jam non móritur, mors illi ultra non dominábitur.~\GreDagger\
Quod enim mórtuus est peccáto, mórtuus est semel:~*
quod autem vivit, vivit Deo.
\Rbar.~Deo grátias.

\section*{Hymn}

\gresetinitiallines{1}
\gregorioscore{../hymni/hymn-aurora-lucis-rutilat}

\begin{versicles}
\Vbar. In resurrectióne tua, Christe, allelúja.

\Rbar. Cæli et terra læténtur, allelúja.
\end{versicles}

\section*{Benedictus}

\textit{The antiphon and tone of the Benedictus are proper to the day.}

\section*{Oration}

\begin{versicles}
\Vbar. Dómine exáudi oratiónem meam.

\Rbar. Et clamor meus ad te véniat.
\end{versicles}

\textit{The oration is proper to the day.}

\Rbar.~Amen.

\textit{Commememorations of occuring feasts are then said if there are any, followed by the Paschaltide commemoration of the Cross. The commemoration of the Cross is ommitted on Low Sunday and within octaves. Each commemoration or suffrage has an antiphon, versicle and response, and prayer. The conclusion of the prayer is omitted for all but the last.}

\subsection*{Commemoration of the Cross}
\gresetinitiallines{1}
\gregorioscore{../suffragia/crucifixus}

\begin{versicles}
\Vbar. Dícite in natiónibus, allelúja.

\Rbar. Quia Dóminus regnávit a ligno, allelúja.
\end{versicles}

Orémus.
Deus qui pro nobis Fílium tuum Crucis patíbulum subíre voluísti, ut inimíci a nobis expélleres potestátem:~*
concéde nobis fámulis tuis; ut resurrectiónis grátiam consequámur.
Per eúndem Dóminum.

\gresetinitiallines{1}
\gregorioscore{../benedicamus/benedicamus-paschaltide}

\textit{In a low voice:}

\begin{versicles}
\Vbar.~Fidelium ánimæ, per misericórdiam Dei, requiéscant in pace.

\Rbar.~Amen.
\end{versicles}

\textit{All say one \textnormal{Pater noster} silently.}

\begin{versicles}
\Vbar. Dóminus det nobis suam pacem.

\Rbar. Et vitam ætérnam. Amen.
\end{versicles}

\section*{Marian Anthem}

\gresetinitiallines{1}
\gregorioscore{../marian-antiphons/regina-caeli-simple}

\begin{versicles}
\Vbar. Gaude et lætáre, Virgo María, allelúja.

\Rbar. Quia surréxit Dóminus vere, allelúja.
\end{versicles}

Orémus.
Deus, qui per resurrectiónem Fílii tui, Dómini nostri Jesu Christi, mundum lætificáre dignátus es:~\GreDagger\
præsta, qu\'{\ae}sumus; ut, per eíus Genetrícem Vírginem Maríam,~*
perpétuæ capiámus gáudia vitæ.
Per eúmdem Christum Dóminum nostrum. \Rbar. Amen.

\begin{versicles}
\Vbar. Divínum auxílium + máneat semper nobíscum.

\Rbar. Amen.
\end{versicles}

\end{document}