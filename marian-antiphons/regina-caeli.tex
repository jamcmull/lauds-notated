% !TEX TS-program = lualatex
% !TEX encoding = UTF-8

\documentclass[11pt, twoside]{report}

\usepackage{fontspec}
\usepackage[utf8]{inputenc}
\usepackage[bitstream-charter]{mathdesign}
\usepackage{bbding}
\usepackage{ragged2e}
\usepackage{parskip}
\usepackage{enumitem}
\usepackage{titlesec}
\usepackage{paracol}
\usepackage{mdframed}
\usepackage[margin=1in]{geometry}

\usepackage[autocompile]{gregoriotex}

\titleformat{\chapter}[block]{\huge\scshape\filcenter}{}{1em}{}
\titleformat{\section}[block]{\Large\bfseries\filcenter}{}{1em}{}

\mdfsetup{skipabove=\topskip, skipbelow=\topskip}

\newenvironment{versicles}{\par\leavevmode\parskip=0pt}{}

\begin{document}

\section*{Regina Cæli (Solemn Tone)}

\textit{This Marian anthem is sung from Compline of Holy Saturday until None of the Saturday within the Octave of Pentecost (inclusive).}

\gresetinitiallines{1}
\gregorioscore{regina-caeli-solemn}

\begin{versicles}
\Vbar. Gaude et lætáre, Virgo María, allelúja.

\Rbar. Quia surréxit Dóminus vere, allelúja.
\end{versicles}

Orémus.
Deus, qui per resurrectiónem Fílii tui, Dómini nostri Jesu Christi, mundum lætificáre dignátus es:~\GreDagger\
præsta, qu\'{\ae}sumus; ut, per eíus Genetrícem Vírginem Maríam,~*
perpétuæ capiámus gáudia vitæ.
Per eúmdem Christum Dóminum nostrum. \Rbar. Amen

\begin{versicles}
\Vbar. Divínum auxílium + máneat semper nobíscum.

\Rbar. Amen.
\end{versicles}

\end{document}