% c4
% 1g2
% 1. Lau(f)dá(gh)te(h) Dó(h)mi(h)<b>num</b>(ixi) de(h) <b>cæ</b>(g)lis:(h.) *(:) lau(h)dá(h)te(h) e(h)um(h) <i>in</i>(g) <i>ex</i>(f)<b>cél</b>(g)sis.(ghg.) <i>Flex:</i>(::) vír(h)gi(g)nes, +(g. h h h  ::)
% 1a
% 1. Lau(f)dá(gh)te(h) Dó(h)mi(h)<b>num</b>(ixi) de(h) <b>cæ</b>(g)lis:(h.) *(:) lau(h)dá(h)te(h) e(h)um(h) <i>in</i>(g) <i>ex</i>(f)<b>cél</b>(g)sis.(h.) <i>Flex:</i>(::) vír(h)gi(g)nes, +(g. h h h  ::)
% 1a2
% 1. Lau(f)dá(gh)te(h) Dó(h)mi(h)<b>num</b>(ixi) de(h) <b>cæ</b>(g)lis:(h.) *(:) lau(h)dá(h)te(h) e(h)um(h) <i>in</i>(g) <i>ex</i>(f)<b>cél</b>(g)sis.(gh..) <i>Flex:</i>(::) vír(h)gi(g)nes, +(g. h h h  ::)

2. Laudáte eum, omnes \textbf{An}geli \textbf{e}jus:~* laudáte eum, omnes vir\textit{tú}\textit{tes} \textbf{e}jus.

3. Laudáte eum, \textbf{sol} et \textbf{lu}na:~* laudáte eum, omnes stel\textit{læ} \textit{et} \textbf{lu}men.

4. Laudáte eum, \textbf{cæ}li cæ\textbf{ló}rum:~* et aquæ omnes, quæ super cælos sunt, laudent \textit{no}\textit{men} \textbf{Dó}mini.

5. Quia ipse \textbf{di}xit, et \textbf{fac}ta sunt:~* ipse mandávit, \textit{et} \textit{cre}\textbf{á}ta sunt.

6. Státuit ea in ætérnum, et in \textbf{s\'{\ae}}culum \textbf{s\'{\ae}}culi:~* præcéptum pósuit, et non \textit{præ}\textit{ter}\textbf{í}bit.

7. Laudáte Dómi\textbf{num} de \textbf{ter}ra,~* dracónes, et om\textit{nes} \textit{a}\textbf{býs}si.

8. Ignis, grando, nix, glácies, spíritus \textbf{pro}cel\textbf{lá}rum:~* quæ fáciunt \textit{ver}\textit{bum} \textbf{e}jus:

9. Montes, et \textbf{om}nes \textbf{col}les:~* ligna fructífera, et \textit{om}\textit{nes} \textbf{ce}dri.

10. Béstiæ, et uni\textbf{vér}sa \textbf{pé}cora:~* serpéntes, et vólu\textit{cres} \textit{pen}\textbf{ná}tæ:

11. Reges terræ, et \textbf{om}nes \textbf{pó}puli:~* príncipes, et omnes jú\textit{di}\textit{ces} \textbf{ter}ræ.

12. Júvenes, et vírgines,~\GreDagger\ senes cum junióribus laudent \textbf{no}men \textbf{Dó}mini:~* quia exaltátum est nomen e\textit{jus} \textit{so}\textbf{lí}us.

13. Conféssio ejus super \textbf{cæ}lum et \textbf{ter}ram:~* et exaltávit cornu pó\textit{pu}\textit{li} \textbf{su}i.

14. Hymnus ómnibus \textbf{sanc}tis \textbf{e}jus:~* fíliis Israël, pópulo appropin\textit{quán}\textit{ti} \textbf{si}bi.

\textbf{Ps. 149.} Cantáte Dómino \textbf{cán}ticum \textbf{no}vum:~* laus ejus in ecclési\textit{a} \textit{sanc}\textbf{tó}rum.

2. Lætétur Israël in eo, qui \textbf{fe}cit \textbf{e}um:~* et fílii Sion exsúltent in \textit{re}\textit{ge} \textbf{su}o.

3. Laudent nomen \textbf{e}jus in \textbf{cho}ro:~* in týmpano, et psaltério \textit{psal}\textit{lant} \textbf{e}i.

4. Quia beneplácitum est Dómino in \textbf{pó}pulo \textbf{su}o:~* et exaltábit mansuétos \textit{in} \textit{sa}\textbf{lú}tem.

5. Exsultábunt \textbf{sanc}ti in \textbf{gló}ria:~* lætabúntur in cubí\textit{li}\textit{bus} \textbf{su}is.

6. Exaltatiónes Dei in gúttu\textbf{re} e\textbf{ó}rum:~* et gládii ancípites in máni\textit{bus} \textit{e}\textbf{ó}rum.

7. Ad faciéndam vindíctam in \textbf{na}ti\textbf{ó}nibus:~* increpatió\textit{nes} \textit{in} \textbf{pó}pulis.

8. Ad alligándos reges eórum \textbf{in} com\textbf{pé}dibus:~* et nóbiles eórum in má\textit{ni}\textit{cis} \textbf{fér}reis.

9. Ut fáciant in eis judíci\textbf{um} con\textbf{scríp}tum:~* glória hæc est ómnibus \textit{sanc}\textit{tis} \textbf{e}jus.

\textbf{Ps. 150.} 1. Laudáte Dóminum in \textbf{sanc}tis \textbf{e}jus:~* laudáte eum in firmaménto vir\textit{tú}\textit{tis} \textbf{e}jus.

2. Laudáte eum in vir\textbf{tú}tibus \textbf{e}jus:~* laudáte eum secúndum multitúdinem magnitú\textit{di}\textit{nis} \textbf{e}jus.

3. Laudáte eum in \textbf{so}no \textbf{tu}bæ:~* laudáte eum in psaltéri\textit{o}, \textit{et} \textbf{cí}thara.

4. Laudáte eum in týmpa\textbf{no}, et \textbf{cho}ro:~* laudáte eum in chor\textit{dis}, \textit{et} \textbf{ór}gano.

5. Laudáte eum in cýmbalis benesonántibus:~\GreDagger\ laudáte eum in cýmbalis jubi\textbf{la}ti\textbf{ó}nis:~* omnis spíritus \textit{lau}\textit{det} \textbf{Dó}minum.

6. Glória \textbf{Pa}tri, et \textbf{Fí}lio,~* et Spirí\textit{tu}\textit{i} \textbf{Sanc}to.

7. Sicut erat in princípio, et \textbf{nunc}, et \textbf{sem}per,~* et in s\'{\ae}cula sæcu\textit{ló}\textit{rum}. \textbf{A}men.