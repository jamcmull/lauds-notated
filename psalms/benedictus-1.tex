% c4
% 1D
% 1. Be(f)ne(gh)díc(h)tus(h) Dó(h)mi(h)nus,(h) <b>De</b>(ixi hr)us(h) <b>Is</b>(g)ra(hr)ël:(h.) *(:) qui(h)a(h) vi(h)si(h)tá(h)vit,(h) et(h) fe(h)cit(h) red(h)emp(h)ti(h)ó(h)nem(h) <i>ple</i>(g)<i>bis</i>(f) <b>su</b>(gh gr)æ:(gvFED.) (::)
% 1D-
% 1. Be(f)ne(gh)díc(h)tus(h) Dó(h)mi(h)nus,(h) <b>De</b>(ixi hr)us(h) <b>Is</b>(g)ra(hr)ël:(h.) *(:) qui(h)a(h) vi(h)si(h)tá(h)vit,(h) et(h) fe(h)cit(h) red(h)emp(h)ti(h)ó(h)nem(h) <i>ple</i>(g)<i>bis</i>(f) <b>su</b>(g gr)æ:(gvFED.) (::)
% 1f
% 1. Be(f)ne(gh)díc(h)tus(h) Dó(h)mi(h)nus,(h) <b>De</b>(ixi hr)us(h) <b>Is</b>(g)ra(hr)ël:(h.) *(:) qui(h)a(h) vi(h)si(h)tá(h)vit,(h) et(h) fe(h)cit(h) red(h)emp(h)ti(h)ó(h)nem(h) <i>ple</i>(g)<i>bis</i>(f) <b>su</b>(gh gr)æ:(gf..) (::)
% 1g
% 1. Be(f)ne(gh)díc(h)tus(h) Dó(h)mi(h)nus,(h) <b>De</b>(ixi hr)us(h) <b>Is</b>(g)ra(hr)ël:(h.) *(:) qui(h)a(h) vi(h)si(h)tá(h)vit,(h) et(h) fe(h)cit(h) red(h)emp(h)ti(h)ó(h)nem(h) <i>ple</i>(g)<i>bis</i>(f) <b>su</b>(gh gr)æ:(g.) (::)
% 1g2
% 1. Be(f)ne(gh)díc(h)tus(h) Dó(h)mi(h)nus,(h) <b>De</b>(ixi hr)us(h) <b>Is</b>(g)ra(hr)ël:(h.) *(:) qui(h)a(h) vi(h)si(h)tá(h)vit,(h) et(h) fe(h)cit(h) red(h)emp(h)ti(h)ó(h)nem(h) <i>ple</i>(g)<i>bis</i>(f) <b>su</b>(g gr)æ:(ghg.) (::)
% 1g3
% 1. Be(f)ne(gh)díc(h)tus(h) Dó(h)mi(h)nus,(h) <b>De</b>(ixi hr)us(h) <b>Is</b>(g)ra(hr)ël:(h.) *(:) qui(h)a(h) vi(h)si(h)tá(h)vit,(h) et(h) fe(h)cit(h) red(h)emp(h)ti(h)ó(h)nem(h) <i>ple</i>(g)<i>bis</i>(f) <b>su</b>(g gr)æ:(g.) (::)
% 1a
% 1. Be(f)ne(gh)díc(h)tus(h) Dó(h)mi(h)nus,(h) <b>De</b>(ixi hr)us(h) <b>Is</b>(g)ra(hr)ël:(h.) *(:) qui(h)a(h) vi(h)si(h)tá(h)vit,(h) et(h) fe(h)cit(h) red(h)emp(h)ti(h)ó(h)nem(h) <i>ple</i>(g)<i>bis</i>(f) <b>su</b>(g hr)æ:(h.) (::)
% 1a2
% 1. Be(f)ne(gh)díc(h)tus(h) Dó(h)mi(h)nus,(h) <b>De</b>(ixi hr)us(h) <b>Is</b>(g)ra(hr)ël:(h.) *(:) qui(h)a(h) vi(h)si(h)tá(h)vit,(h) et(h) fe(h)cit(h) red(h)emp(h)ti(h)ó(h)nem(h) <i>ple</i>(g)<i>bis</i>(f) <b>su</b>(g gr)æ:(gh..) (::)
% 1a3
% 1. Be(f)ne(gh)díc(h)tus(h) Dó(h)mi(h)nus,(h) <b>De</b>(ixi hr)us(h) <b>Is</b>(g)ra(hr)ël:(h.) *(:) qui(h)a(h) vi(h)si(h)tá(h)vit,(h) et(h) fe(h)cit(h) red(h)emp(h)ti(h)ó(h)nem(h) <i>ple</i>(g)<i>bis</i>(f) <b>su</b>(gh gr)æ:(gh..) (::)

2. Et eréxit cornu sa\textbf{lú}tis \textbf{no}bis:~* in domo David, pú\textit{e}\textit{ri} \textbf{su}i.

3. Sicut locútus est per \textbf{os} sanc\textbf{tó}rum,~* qui a s\'{\ae}culo sunt, prophe\textit{tá}\textit{rum} \textbf{e}jus:

4. Salútem ex ini\textbf{mí}cis \textbf{nos}tris,~* et de manu ómnium, \textit{qui} \textit{o}\textbf{dé}runt nos.

5. Ad faciéndam misericórdiam cum \textbf{pá}tribus \textbf{nos}tris:~* et memorári testaménti \textit{su}\textit{i} \textbf{sanc}ti.

6. Jusjurándum, quod jurávit ad Abraham \textbf{pa}trem \textbf{nos}trum,~* datú\textit{rum} \textit{se} \textbf{no}bis:

7. Ut sine timóre, de manu inimicórum nostrórum \textbf{li}be\textbf{rá}ti,~* servi\textit{á}\textit{mus} \textbf{il}li.

8. In sanctitáte, et justítia \textbf{co}ram \textbf{ip}so,~* ómnibus di\textit{é}\textit{bus} \textbf{nos}tris.

9. Et tu, puer, Prophéta Altíssi\textbf{mi} vo\textbf{cá}beris:~* præíbis enim ante fáciem Dómini, paráre \textit{vi}\textit{as} \textbf{e}jus:

10. Ad dandam sciéntiam salútis \textbf{ple}bi \textbf{e}jus:~* in remissiónem peccató\textit{rum} \textit{e}\textbf{ó}rum:

11. Per víscera misericórdiæ \textbf{De}i \textbf{nos}tri:~* in quibus visitávit nos, óri\textit{ens} \textit{ex} \textbf{al}to:

12. Illumináre his, qui in ténebris, et in umbra \textbf{mor}tis \textbf{se}dent:~* ad dirigéndos pedes nostros in \textit{vi}\textit{am} \textbf{pa}cis.

13. Glória \textbf{Pa}tri, et \textbf{Fí}lio,~* et Spirí\textit{tu}\textit{i} \textbf{Sanc}to.

14. Sicut erat in princípio, et \textbf{nunc}, et \textbf{sem}per,~* et in s\'{\ae}cula sæcu\textit{ló}\textit{rum}. \textbf{A}men.