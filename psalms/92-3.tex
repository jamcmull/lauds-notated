% c4
% 3b antiquo
% 1. Dó(g)mi(hi)nus(i) re(i)gná(i)vit,(i) de(i)<b>có</b>(k)rem(jr) ind(j)<b>ú</b>(jr[ocb:1{])<b>tus</b>(ih[ocb:0}]) est:(j.) *(:) ind(i)ú(i)tus(i) est(i) Dó(i)mi(i)nus(i) for(i)ti(i)tú(i)di(i)nem,(i) <b>et</b>(j hr) præ(h)<b>cín</b>(j)xit(jr) se.(i.) (::)
% 3a antiquo
% 1. Dó(g)mi(hi)nus(i) re(i)gná(i)vit,(i) de(i)<b>có</b>(k)rem(jr) ind(j)<b>ú</b>(jr[ocb:1{])<b>tus</b>(ih[ocb:0}]) est:(j.) *(:) ind(i)ú(i)tus(i) est(i) Dó(i)mi(i)nus(i) for(i)ti(i)tú(i)di(i)nem,(i) <b>et</b>(j hr) præ(h)<b>cín</b>(j)xit(jr) se.(ih..) (::)
% 3a2 antiquo
% 1. Dó(g)mi(hi)nus(i) re(i)gná(i)vit,(i) de(i)<b>có</b>(k)rem(jr) ind(j)<b>ú</b>(jr[ocb:1{])<b>tus</b>(ih[ocb:0}]) est:(j.) *(:) ind(i)ú(i)tus(i) est(i) Dó(i)mi(i)nus(i) for(i)ti(i)tú(i)di(i)nem,(i) <b>et</b>(j hr) præ(hi)<b>cín</b>(h)xit(gr) se.(gh..) (::)
% 3g antiquo
% 1. Dó(g)mi(hi)nus(i) re(i)gná(i)vit,(i) de(i)<b>có</b>(k)rem(jr) ind(j)<b>ú</b>(jr[ocb:1{])<b>tus</b>(ih[ocb:0}]) est:(j.) *(:) ind(i)ú(i)tus(i) est(i) Dó(i)mi(i)nus(i) for(i)ti(i)tú(i)di(i)nem,(i) <b>et</b>(j hr) præ(hi)<b>cín</b>(h)xit(gr) se.(g.) (::)

2. Etenim firmávit \textbf{or}bem \textbf{ter}ræ,~* qui non \textbf{com}mo\textbf{vé}bitur.

3. Paráta sedes \textbf{tu}a \textbf{ex} tunc:~* a \textbf{s\'{\ae}}culo \textbf{tu} es.

4. Elevavérunt \textbf{flú}mina, \textbf{Dó}\textbf{mi}ne:~* elevavérunt flúmina \textbf{vo}cem \textbf{su}am.

5. Elevavérunt flúmina \textbf{fluc}tus \textbf{su}os,~* a vócibus a\textbf{quá}rum mul\textbf{tá}rum.

6. Mirábiles elati\textbf{ó}nes \textbf{ma}ris:~* mirábilis in \textbf{al}tis \textbf{Dó}minus.

7. Testimónia tua credibília \textbf{fac}ta sunt \textbf{ni}mis:~* domum tuam decet sanctitúdo, Dómine, in longitúdi\textbf{nem} di\textbf{é}rum.

8. Glória \textbf{Pa}tri, et \textbf{Fí}\textbf{li}o,~* et Spi\textbf{rí}tui \textbf{Sanc}to.

9. Sicut erat in princípio, et \textbf{nunc}, et \textbf{sem}per,~* et in s\'{\ae}cula sæcu\textbf{ló}rum. \textbf{A}men.