% !TEX TS-program = lualatex
% !TEX encoding = UTF-8

\documentclass[11pt, twoside]{report}

\usepackage{fontspec}
\usepackage[utf8]{inputenc}
\usepackage[bitstream-charter]{mathdesign}
\usepackage{bbding}
\usepackage{ragged2e}
\usepackage{parskip}
\usepackage{enumitem}
\usepackage{titlesec}
\usepackage{paracol}
\usepackage{mdframed}
\usepackage[margin=1in]{geometry}

\usepackage[autocompile]{gregoriotex}

\titleformat{\chapter}[block]{\huge\scshape\filcenter}{}{1em}{}
\titleformat{\section}[block]{\Large\bfseries\filcenter}{}{1em}{}

\mdfsetup{skipabove=\topskip, skipbelow=\topskip}

\newenvironment{versicles}{\par\leavevmode\parskip=0pt}{}

\begin{document}

\chapter*{Sunday within the Octave of Corpus Christ\\19 June 2022}

\textit{Omit the conclusion \textnormal{Qui vivis et regnas} from the commemoration of the Octave.}

\subsection*{Commemoration of St. Juliana Falconieri, Virgin}

\gresetinitiallines{1}
\gregorioscore{../commemorations/one-virgin}

\begin{versicles}
\Vbar. Diffúsa est grátia in lábiis tuis.

\Rbar. Proptérea benedíxit te Deus in ætérnum.
\end{versicles}

Orémus.
Deus, qui beátam Juliánam Vírginem tuam extrémo morbo laborántem, pretióso Fílii tui Córpore mirabíliter recreáre dignátus es:~\GreDagger\
concéde, qu\'{\ae}sumus; ut, ejus intercedéntibus méritis, nos quoque eódem in mortis agóne refécti ac roboráti,~*
ad cæléstem pátriam perducámur.

\subsection*{Commemoration of Ss. Saints Gervase and Protase, Martyrs}

\gresetinitiallines{1}
\gregorioscore{../commemorations/many-martyrs}

\begin{versicles}
\Vbar. Exsultábunt Sancti in gloria.

\Rbar. Lætabúntur in cubílibus suis.
\end{versicles}

Orémus.
Deus, qui nos ánnua sanctórum Mártyrum tuórum Gervásii et Protásii solemnitáte lætíficas:~*
concéde propítius; ut, quorum gaudémus méritis, accendámur exémplis.
Per Dóminum

\end{document}